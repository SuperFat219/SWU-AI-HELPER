% -------------------------允许算法跨页-------------
\makeatletter
\newenvironment{breakablealgorithm}
{% \begin{breakablealgorithm}
	\begin{center}
		\refstepcounter{algorithm}% New algorithm
		\hrule height.8pt depth0pt \kern2pt% \@fs@pre for \@fs@ruled
		\renewcommand{\caption}[2][\relax]{% Make a new \caption
			{\raggedright\textbf{\ALG@name~\thealgorithm} ##2\par}%
			\ifx\relax##1\relax % #1 is \relax
			\addcontentsline{loa}{algorithm}{\protect\numberline{\thealgorithm}##2}%
			\else % #1 is not \relax
			\addcontentsline{loa}{algorithm}{\protect\numberline{\thealgorithm}##1}%
			\fi
			\kern2pt\hrule\kern2pt
		}
	}{% \end{breakablealgorithm}
		\kern2pt\hrule\relax% \@fs@post for \@fs@ruled
	\end{center}
}
\makeatother

%----------------------------------------------------
\documentclass[11pt,a4paper]{ctexart}
\usepackage{fontspec}
\usepackage{xeCJK}%文字体的设置
%\newcommand{\canger}{\CJKfontspec{TsangerJinKai03}}
%\setmainfont{TsangerJinKai03} %设置全文字体
%\setsansfont{TsangerJinKai03}
%\setsansfont{TsangerJinKai03} 
%\defaultfontfeatures{Mapping=tex-text}
%\setCJKmainfont{TsangerJinKai03}
%\setCJKsansfont{TsangerJinKai03}
\usepackage{xunicode}
\usepackage{xltxtra}
\usepackage{amsmath}
\usepackage{amsfonts}
\usepackage{amssymb}
\usepackage{graphicx}
\usepackage{amsthm}
\usepackage{array}
\usepackage{tcolorbox}
\usepackage{float}   %{H}
\usepackage{booktabs}  %\toprule[1.5pt]
\usepackage[titletoc]{appendix}
\usepackage{algorithm}  
\usepackage{algorithmic}  
%\usepackage{algorithmicx}
%===================%插入代码需要的控制
\usepackage{listings}
\usepackage{xcolor}
\setmonofont{Consolas}%字体
\lstset{%
	numbers=left,
	numberstyle=\tt\tiny,%
	showstringspaces=false,
	showspaces=false,%
	tabsize=4,%
	frame=lines,%
	basicstyle=\tt\small,%
	keywordstyle=\color{ blue!70}\bfseries,%
	identifierstyle=,%
	commentstyle=\color{red!50!green!50!blue!50},%\itshape,%
	stringstyle=\color{black},%
	breaklines=true
}
%===================%
\usepackage[left=2cm,right=2cm,top=2cm,bottom=2cm]{geometry}
\newtheorem*{solution}{解}
\newtheorem{cor}{推论}
\newtheorem{e}{习题}
\title{算法设计与分析(5.20作业)}
\author{智科三班 \quad 严中圣 \quad 222020335220177}
\date{\today}
\begin{document}
\maketitle
\pagestyle{plain}%设置页码
%\noindent\rule{\textwidth}{1pt} %加横线
%============================================================================
\newpage
\begin{breakablealgorithm}
	\caption{$0-1 \; Knapsack problem\; ——\; Brute \; Force $} 
	\begin{algorithmic}[1]
		\REQUIRE $n,c,V[n],W[n]$
		\ENSURE $X[n]$ 最终存储状态;$MaxValue$ 最大价值
		\STATE $Initialize$ $X[n],S[n]$ //最终存储状态,当前存储状态
		\STATE $Initialize$ $CurrentValue,CurrentVolume$ //当前价值,当前消耗容量
		\STATE $MaxValue \gets$ $Force(1)$
		\RETURN $X[n],MaxValue$
	\end{algorithmic}
\end{breakablealgorithm}

\begin{breakablealgorithm}
	\caption{$Force(i)$} 
	\begin{algorithmic}[1]
%		\REQUIRE $n,c,V[n],W[n]$
%		\ENSURE $X[n]$ 最终存储状态;$MaxValue$ 最大价值
%		\STATE $Initialize$ $X[n],S[n]$ //最终存储状态,当前存储状态
%		\STATE $Initialize$ $CurrentValue,CurrentVolumn$ //当前价值,当前消耗容量
		\IF {$i>n$}
			\IF {$MaxValue < CurrentValue \quad and \quad  CurrentVolumn \leq c$}
				\FOR {$k \gets 1 \; \textbf{to} \; n$}
					\STATE $X[k] \gets S[k]$ //存储最优路径
				\ENDFOR
				\STATE $MaxValue \gets CurrentValue$
			\ENDIF
			\RETURN $MaxValue$
		\ENDIF
		\STATE $CurrentVolume \gets$ $CurrentVolume + V[i]$
		\STATE $CurrentValue \gets$ $CurrentValue + W[i]$
		\STATE $S[i]=1$
		\STATE $Force(i+1)$
		\STATE $CurrentVolume \gets$ $CurrentVolume - V[i]$
		\STATE $CurrentValue \gets$ $CurrentValue - W[i]$
		\STATE $S[i] \gets 0$
		\STATE $Force(i+1)$
		\RETURN $MaxValue$
	\end{algorithmic}
\end{breakablealgorithm}


\newpage
\begin{breakablealgorithm}
	\caption{$0-1 \; Knapsack problem\; ——\; Dynamic \; processing$} 
	\begin{algorithmic}[1]
		\REQUIRE $n,c,V[n],W[n]$
		\ENSURE $X[n]$ 最终存储状态;$MaxValue$ 最大价值
		\STATE $Initialize$ $X[n]$ //最终存储状态,当前存储状态
		\STATE $Initialize$ $V[n+1][c]$ //dp数组,初始化数组为0
		\FOR {$k \gets 1 \; \textbf{to} \; c$}
			\IF {$k>V[1]$}
				\STATE $V[1][k]\gets W[1]$
			\ENDIF
		\ENDFOR
		\FOR {$i \gets 2 \; \textbf{to} \; n$}
			\FOR{$j \gets 1 \; \textbf{to} \; c$}
				\IF {$j<V[i-1]$}
					\STATE $V[i][j] \gets  V[i-1][j]$
				\ELSE 
					\STATE $V[i][j] \gets \max(V[i-1][j],V[i-1][j-V[i-1]]+W[i-1])$
				\ENDIF
			\ENDFOR
		\ENDFOR
		\STATE $j \gets c$
		\FOR {$i \gets n \; \textbf{to} \; 1$}
			\IF {$i == 1$}
				\IF {$V[i][j]>0$}
					\STATE $X[1] \gets 1$
				\ELSE
					\STATE $X[1] \gets 0$
				\ENDIF
			\ENDIF
			\IF {$V[i][j] > V[i-1][j]$}
				\STATE $X[i] \gets 1$
				\STATE $j \gets j - V[i]$
			\ELSE
				\STATE $X[i] \gets 0$
			\ENDIF
		\ENDFOR
		\RETURN $X[n],V[n][c]$
	\end{algorithmic}
\end{breakablealgorithm}

\newpage 
\begin{breakablealgorithm}
	\caption{$0-1 \; Knapsack problem\; ——\; BackTracking \; Method $} 
	\begin{algorithmic}[1]
		\REQUIRE $n,c,V[n],W[n]$
		\ENSURE $X[n]$ 最终存储状态;$MaxValue$ 最大价值
		\STATE $Initialize$ $X[n],S[n]$ //最终存储状态,当前存储状态
		\STATE $Initialize$ $CurrentValue,CurrentVolume$ //当前价值,当前消耗容量
		\STATE $MaxValue \gets$ $BackTrack(1)$
		\RETURN $X[n],MaxValue$
	\end{algorithmic}
\end{breakablealgorithm}

\begin{breakablealgorithm}
	\caption{$BackTrack(i)$} 
	\begin{algorithmic}[1]
		%		\REQUIRE $n,c,V[n],W[n]$
		%		\ENSURE $X[n]$ 最终存储状态;$MaxValue$ 最大价值
		%		\STATE $Initialize$ $X[n],S[n]$ //最终存储状态,当前存储状态
		%		\STATE $Initialize$ $CurrentValue,CurrentVolumn$ //当前价值,当前消耗容量
		\IF {$i>n$}
			\IF {$MaxValue < CurrentValue$}
				\FOR {$k \gets 1 \; \textbf{to} \; n$}
					\STATE $X[k] \gets S[k]$ //存储最优路径
				\ENDFOR
				\STATE $MaxValue \gets CurrentValue$
			\ENDIF
			\RETURN $MaxValue$
		\ENDIF
		\IF {$CurrentVolume + V[i] \leq c$}
			\STATE $CurrentVolume \gets$ $CurrentVolume + V[i]$
			\STATE $CurrentValue \gets$ $CurrentValue + W[i]$
			\STATE $S[i]\gets 1$
			\STATE $BackTrack(i+1)$
			\STATE $CurrentVolume \gets$ $CurrentVolume - V[i]$
			\STATE $CurrentValue \gets$ $CurrentValue - W[i]$
		\ENDIF
		\STATE $S[i]\gets 0$
		\STATE $BackTrack(i+1)$
		\RETURN $MaxValue$
	\end{algorithmic}
\end{breakablealgorithm}


\newpage
\begin{breakablealgorithm}
	\caption{$0-1 \; Knapsack problem\; ——\; Branch \;and \;Bound $} 
	\begin{algorithmic}[1]
		\REQUIRE $n,c,V[n],W[n]$
		\ENSURE $X[n]$ 最终存储状态;$MaxValue$ 最大价值
		\STATE $Initialize$ $X[n],S[n]$ //最终存储状态,当前存储状态
		\STATE $Initialize$ $CurrentValue,CurrentVolume$ //当前价值,当前消耗容量
		\STATE $MaxValue \gets$ $BackTrack(1)$
		\RETURN $X[n],MaxValue$
	\end{algorithmic}
\end{breakablealgorithm}

\begin{breakablealgorithm}
	\caption{$BackTrack(i)$} 
	\begin{algorithmic}[1]
		%		\REQUIRE $n,c,V[n],W[n]$
		%		\ENSURE $X[n]$ 最终存储状态;$MaxValue$ 最大价值
		%		\STATE $Initialize$ $X[n],S[n]$ //最终存储状态,当前存储状态
		%		\STATE $Initialize$ $CurrentValue,CurrentVolumn$ //当前价值,当前消耗容量
		\IF {$i>n$}
			\IF {$MaxValue < CurrentValue$}
				\FOR {$k \gets 1 \; \textbf{to} \; n$}
					\STATE $X[k] \gets S[k]$ //存储最优路径
				\ENDFOR
				\STATE $MaxValue \gets CurrentValue$
			\ENDIF
			\RETURN
		\ENDIF
		\IF {$CurrentVolume + V[i] \leq c$}
			\STATE $CurrentVolume \gets$ $CurrentVolume + V[i]$
			\STATE $CurrentValue \gets$ $CurrentValue + W[i]$
			\STATE $BackTrack(i+1)$
			\STATE $CurrentVolume \gets$ $CurrentVolume - V[i]$
			\STATE $CurrentValue \gets$ $CurrentValue - W[i]$
		\ENDIF
		\IF {$Bound(i+1) > MaxValue$}
			\STATE $S[i]\gets 0$
			\STATE $BackTrack(i+1)$
		\ENDIF
	\end{algorithmic}
\end{breakablealgorithm}
\begin{breakablealgorithm}
	\caption{$Bound(i)$} 
	\begin{algorithmic}[1]
		\STATE $Initialize$ $RemainValue$ //剩余最大价值
		\WHILE {$k \leq n$}
			\STATE $RemainValue \gets RemainValue+W[k]$
			\STATE $k\gets k+1$
		\ENDWHILE
		\RETURN $RemainValue+CurrentValue$
	\end{algorithmic}
\end{breakablealgorithm}
\end{document}